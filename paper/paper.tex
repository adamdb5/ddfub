\documentclass[a4paper]{article}
\usepackage[margin=1in]{geometry}
\usepackage{hyperref}
\usepackage{listings}

\hypersetup{colorlinks=false, linktoc=all}
\usepackage{xcolor}
\usepackage{textcomp}

\definecolor{comment}{RGB}{0,128,0} % dark green
\definecolor{string}{RGB}{255,0,0}  % red
\definecolor{keyword}{RGB}{0,0,255} % blue

\lstdefinestyle{c}{
	commentstyle=\color{comment},
	stringstyle=\color{string},
	keywordstyle=\color{keyword},
	basicstyle=\footnotesize\ttfamily,
	numbers=left,
	numberstyle=\tiny,
	numbersep=5pt,
	frame=lines,
	breaklines=true,
	prebreak=\raisebox{0ex}[0ex][0ex]{\ensuremath{\hookleftarrow}},
	showstringspaces=false,
	upquote=true,
	tabsize=2,
}

\lstdefinestyle{make}{
	commentstyle=\color{comment},
	stringstyle=\color{string},
	keywordstyle=\color{keyword},
	basicstyle=\footnotesize\ttfamily,
	numbers=left,
	numberstyle=\tiny,
	numbersep=5pt,
	frame=lines,
	breaklines=true,
	prebreak=\raisebox{0ex}[0ex][0ex]{\ensuremath{\hookleftarrow}},
	showstringspaces=false,
	upquote=true,
	tabsize=2,
}


\title{Using Blockchain to Create a Decentralised Security Model for Distributed Systems}
\date{February 2021}
\author{Adam Bruce \\ \texttt{\href{mailto:a.bruce3@ncl.ac.uk}{a.bruce3@ncl.ac.uk}}}
\begin{document}
\maketitle

\begin{abstract}

\end{abstract}

\newpage
\tableofcontents
\newpage

\section{Introduction}
\subsection{Distributed Security}
The problem of distributed security has recently become more important then ever with the rapidly increasing number of Internet of Things (IoT) devices.

\subsection{Univerity Cyberattacks}
During the midst of the COVID-19 pandemic in 2020, several organisations in addition to numerous research and educational institutions were hit by cyberattacks. Two of these universities were Newcastle University and Northumbria University, and due to both the geographical proximity of these universities and the short interval between the attacks provides a strong ground to believe that the same actor used the same attack on both universities.
The apparent lack of communication between universities that were under attack appeared to contribute largely to the repeated effectiveness of such attacks, which inspired me to investigate how we could create a decentralised system to allow universities to share both attempted and successful cyberattacks, allowing other universities to configure their firewalls accordingly, enabling them to successfully defend against these now known attack vectors.

\subsection{Aim}
The original aim of this project was to create a decentralised firewall for distributed systems using blockchain, however throughout my preliminary investigation, I learned that there was no existing public research for any implementation or framework for establishing decentralised security.
Consequently, I decided that I would adapt my project to research how we could use blockchain to create a general model for decentralised security, and provide an implementation of such a system in the C programming language. This allowed me to focus primarily on designing a model for decentralised security, instead of creating a firewall, which is already a well established area of research.

\subsection{Objectives}

\section{Background Material}
background.

\section{What I did and why}



% Example code

blockchain.h
\lstinputlisting[language={[ANSI]c},style=c]{../src/blockchain.h}

blockchain.c
\lstinputlisting[language={[ANSI]c},style=c]{../src/blockchain.c}

Makefile
\lstinputlisting[language=make,style=make]{../src/Makefile}
%%


\end{document}