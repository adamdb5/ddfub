\documentclass[a4paper]{report}
\usepackage[margin=1in]{geometry}
\usepackage{hyperref}
\usepackage{listings}
\usepackage[strings]{underscore}

\hypersetup{colorlinks=false, linktoc=all}
\usepackage{xcolor}

\definecolor{comment}{RGB}{0,128,0} % dark green
\definecolor{string}{RGB}{255,0,0}  % red
\definecolor{keyword}{RGB}{0,0,255} % blue

\lstdefinestyle{c}{
	commentstyle=\color{comment},
	stringstyle=\color{string},
	keywordstyle=\color{keyword},
	basicstyle=\footnotesize\ttfamily,
	numbers=left,
	numberstyle=\tiny,
	numbersep=5pt,
	frame=lines,
	breaklines=true,
	prebreak=\raisebox{0ex}[0ex][0ex]{\ensuremath{\hookleftarrow}},
	showstringspaces=false,
	upquote=true,
	tabsize=2,
}

\lstdefinestyle{make}{
	commentstyle=\color{comment},
	stringstyle=\color{string},
	keywordstyle=\color{keyword},
	basicstyle=\footnotesize\ttfamily,
	numbers=left,
	numberstyle=\tiny,
	numbersep=5pt,
	frame=lines,
	breaklines=true,
	prebreak=\raisebox{0ex}[0ex][0ex]{\ensuremath{\hookleftarrow}},
	showstringspaces=false,
	upquote=true,
	tabsize=2,
}


\title{Using Blockchain to Create a Decentralised Security Model for Distributed Systems}
\date{April 2021}
\author{Adam Bruce \\ \texttt{\href{mailto:a.bruce3@ncl.ac.uk}{a.bruce3@ncl.ac.uk}}}
\begin{document}
\maketitle

\begin{abstract}
//TODO
\end{abstract}

\tableofcontents

\newpage

\chapter{Introduction}
\section{University Cyberattacks}
In the summer of 2020 during the midst of the COVID-19 pandemic, universities and research institutions worldwide were working hard to understand the structure of the virus and develop a vaccine in an attempt to return to normality. However, whereas some countries were making fast progress in understanding the virus, others were falling behind, and the virus began to put a strain on healthcare, and increasing critique on governments. In order to keep up with the nations at the forefront of vaccine development, nations turned to state-sponsored cyberattacks in order to both hinder nations, and also obtain research and information about other countries' vaccine efforts. One such example was the threat group 'Cozy Bear', formally known as Advanced Persistent Threat (APT) 29. APT29 used a number of tools to target various organisations involved in COVID-19 vaccine development in Canada, the United States and the United Kingdom. The NCSC believe that the intention was highly likely stealing information and intellectual property relating to the vaccine \cite{APT29}.

In addition to the mortality of COVID-19, the virus also caused a number of economic issues across a number of nations. Global stock markets lost \$6 trillion in value over size days from 23 to 28 February \cite{covspill}. This gave private companies no other choice than to make large volumes of staff redundant, which increased job insecurity causing many people to become redundant, and in nations without suitable support or benefits, attackers turned to cybercrime for financial gain. These attacks represented the majority of cyberattacks aimed at both universities and the general public. A study of cyber-crime throughout the COVID-19 pandemic determined that 34\% of attacks directly involved financial fraud with a number of attack surfaces used, the majority being phishing, smishing and malware \cite{diffattack}.

University attacks became a frequent headline in the UK as universities suffered attacks from different threat actors. A number of threat actors launched attacks against multiple universities in the hope to find a vulnerability in at least one. One such attack was aimed at both Newcastle University and Northumbria University, two universities in extremely close proximity \cite{newhack,norhack}. The attack crippled both Newcastle and Northumbria Universities, however the attackers only managed to exfiltrate data from Newcastle University. Why was the attack successful on both occasions? Why wasn't knowledge of the attack shared? 

One reason is that currently, there is no reliable or automated system in place to share this information. Such a system is what this paper will aim to create. 

\chapter{Background Material}

\section{Distributed Systems}
A distributed system consists of a collection 



\section{Aim}
The original aim of this project was to create a decentralised firewall for distributed systems using blockchain, however throughout my preliminary investigation, I learned that there was no existing public research for any implementation or framework for establishing decentralised security.
Consequently, I decided that I would adapt my project to research how we could use blockchain to create a general model for decentralised security, and provide an implementation of such a system in the C programming language. This allowed me to focus primarily on designing a model for decentralised security, instead of creating a firewall, which is already a well established area of research.

\section{Objectives}
During the midst of the COVID-19 pandemic in 2020, several organisations in addition to numerous research and educational institutions were hit by cyberattacks. Two of these universities were Newcastle University and Northumbria University, and due to both the geographical proximity of these universities and the short interval between the attacks provides a strong ground to believe that the same actor used the same attack on both universities.
The apparent lack of communication between universities that were under attack appeared to contribute largely to the repeated effectiveness of such attacks, which inspired me to investigate how we could create a decentralised system to allow universities to share both attempted and successful cyberattacks, allowing other universities to configure their firewalls accordingly, enabling them to successfully defend against these now known attack vectors.

During the midst of the COVID-19 pandemic in 2020, several organisations in addition to numerous research and educational institutions were hit by cyberattacks. Two of these universities were Newcastle University and Northumbria University, and due to both the geographical proximity of these universities and the short interval between the attacks provides a strong ground to believe that the same actor used the same attack on both universities.
The apparent lack of communication between universities that were under attack appeared to contribute largely to the repeated effectiveness of such attacks, which inspired me to investigate how we could create a decentralised system to allow universities to share both attempted and successful cyberattacks, allowing other universities to configure their firewalls accordingly, enabling them to successfully defend against these now known attack vectors.

During the midst of the COVID-19 pandemic in 2020, several organisations in addition to numerous research and educational institutions were hit by cyberattacks. Two of these universities were Newcastle University and Northumbria University, and due to both the geographical proximity of these universities and the short interval between the attacks provides a strong ground to believe that the same actor used the same attack on both universities.
The apparent lack of communication between universities that were under attack appeared to contribute largely to the repeated effectiveness of such attacks, which inspired me to investigate how we could create a decentralised system to allow universities to share both attempted and successful cyberattacks, allowing other universities to configure their firewalls accordingly, enabling them to successfully defend against these now known attack vectors.


background.

\chapter{What I did and why}

\bibliography{bib} 
\bibliographystyle{ieeetr}


\end{document}